% Options for packages loaded elsewhere
\PassOptionsToPackage{unicode}{hyperref}
\PassOptionsToPackage{hyphens}{url}
%
\documentclass[
]{book}
\usepackage{lmodern}
\usepackage{amssymb,amsmath}
\usepackage{ifxetex,ifluatex}
\ifnum 0\ifxetex 1\fi\ifluatex 1\fi=0 % if pdftex
  \usepackage[T1]{fontenc}
  \usepackage[utf8]{inputenc}
  \usepackage{textcomp} % provide euro and other symbols
\else % if luatex or xetex
  \usepackage{unicode-math}
  \defaultfontfeatures{Scale=MatchLowercase}
  \defaultfontfeatures[\rmfamily]{Ligatures=TeX,Scale=1}
\fi
% Use upquote if available, for straight quotes in verbatim environments
\IfFileExists{upquote.sty}{\usepackage{upquote}}{}
\IfFileExists{microtype.sty}{% use microtype if available
  \usepackage[]{microtype}
  \UseMicrotypeSet[protrusion]{basicmath} % disable protrusion for tt fonts
}{}
\makeatletter
\@ifundefined{KOMAClassName}{% if non-KOMA class
  \IfFileExists{parskip.sty}{%
    \usepackage{parskip}
  }{% else
    \setlength{\parindent}{0pt}
    \setlength{\parskip}{6pt plus 2pt minus 1pt}}
}{% if KOMA class
  \KOMAoptions{parskip=half}}
\makeatother
\usepackage{xcolor}
\IfFileExists{xurl.sty}{\usepackage{xurl}}{} % add URL line breaks if available
\IfFileExists{bookmark.sty}{\usepackage{bookmark}}{\usepackage{hyperref}}
\hypersetup{
  pdftitle={金融数学},
  pdfauthor={Financial Mathematics},
  hidelinks,
  pdfcreator={LaTeX via pandoc}}
\urlstyle{same} % disable monospaced font for URLs
\usepackage{longtable,booktabs}
% Correct order of tables after \paragraph or \subparagraph
\usepackage{etoolbox}
\makeatletter
\patchcmd\longtable{\par}{\if@noskipsec\mbox{}\fi\par}{}{}
\makeatother
% Allow footnotes in longtable head/foot
\IfFileExists{footnotehyper.sty}{\usepackage{footnotehyper}}{\usepackage{footnote}}
\makesavenoteenv{longtable}
\usepackage{graphicx,grffile}
\makeatletter
\def\maxwidth{\ifdim\Gin@nat@width>\linewidth\linewidth\else\Gin@nat@width\fi}
\def\maxheight{\ifdim\Gin@nat@height>\textheight\textheight\else\Gin@nat@height\fi}
\makeatother
% Scale images if necessary, so that they will not overflow the page
% margins by default, and it is still possible to overwrite the defaults
% using explicit options in \includegraphics[width, height, ...]{}
\setkeys{Gin}{width=\maxwidth,height=\maxheight,keepaspectratio}
% Set default figure placement to htbp
\makeatletter
\def\fps@figure{htbp}
\makeatother
\setlength{\emergencystretch}{3em} % prevent overfull lines
\providecommand{\tightlist}{%
  \setlength{\itemsep}{0pt}\setlength{\parskip}{0pt}}
\setcounter{secnumdepth}{5}
\usepackage{booktabs}
\usepackage{ctex}
\usepackage{amsthm}
\makeatletter
\def\thm@space@setup{%
  \thm@preskip=8pt plus 2pt minus 4pt
  \thm@postskip=\thm@preskip
}
\makeatother
\usepackage[]{natbib}
\bibliographystyle{apalike}

\title{金融数学}
\author{Financial Mathematics}
\date{2020-09-10 13:00:10}

\begin{document}
\maketitle

{
\setcounter{tocdepth}{1}
\tableofcontents
}
\hypertarget{section}{%
\chapter*{}\label{section}}
\addcontentsline{toc}{chapter}{}

欢迎来到\textbf{金融数学}!

在这里,我们同步课堂,总结每章的\textbf{重点、难点},并发布\textbf{课后作业}。课后作业需在下次上课前交到老师信箱(明主1036门外邮箱柜右下角)。

我们这里主要以英文表述,有以下两个原因

\begin{enumerate}
\def\labelenumi{\arabic{enumi}.}
\item
  方便大家准备SOA/CAS的 \href{https://www.soa.org/education/exam-req/edu-exam-fm-detail/}{Exam FM: Financial Mathematics}考试;
\item
  方便大家阅读相关英文文献。
\end{enumerate}

此网站由授课老师高光远、助教程轶鹏、助教胡夏新管理,欢迎大家反馈意见到助教、微信群、或邮箱\href{mailto:guangyuan.gao@ruc.edu.cn}{\nolinkurl{guangyuan.gao@ruc.edu.cn}}。

\hypertarget{interest-rate}{%
\chapter{Interest rate}\label{interest-rate}}

\hypertarget{key-concepts}{%
\section{Key concepts}\label{key-concepts}}

\hypertarget{functions}{%
\subsection*{Functions}\label{functions}}
\addcontentsline{toc}{subsection}{Functions}

\begin{itemize}
\item
  Accumulation function \[a(t)\]
\item
  Discount function \[a^{-1}(t)\]
\end{itemize}

\hypertarget{interest-rate-1}{%
\subsection*{Interest rate}\label{interest-rate-1}}
\addcontentsline{toc}{subsection}{Interest rate}

\begin{itemize}
\item
  Effective rate of interest/discount \[i,d\]
\item
  Simple interest \[a(t)=1+it\]
\item
  Compound interest \[a(t)=(1+i)^t\]
\item
  Discount factor \[v=(1+i)^{-1}\]
\item
  Accumulation factor of \(t\) years \[(1+i)^t\]
\item
  Discount factor of \(t\) years \[(1+i)^{-t}\]
\item
  Nominal rate of interest/discount \[i^{(m)},d^{(m)}\]
\item
  Force of interest \[\delta\]
\end{itemize}

\hypertarget{values}{%
\subsection*{Values}\label{values}}
\addcontentsline{toc}{subsection}{Values}

\begin{itemize}
\item
  Accumulated value (future value)
\item
  Present value
\end{itemize}

\hypertarget{key-equations}{%
\section{Key equations}\label{key-equations}}

\hypertarget{interest-rate-and-discount-rate}{%
\subsection*{Interest rate and discount rate}\label{interest-rate-and-discount-rate}}
\addcontentsline{toc}{subsection}{Interest rate and discount rate}

\[i=\frac{d}{1-d}\]

\[d=\frac{i}{1+i}\]

\[d=iv\]

\[v=1-d\]

\[i-d=id\]

\hypertarget{accumulation-and-discount}{%
\subsection*{Accumulation and discount}\label{accumulation-and-discount}}
\addcontentsline{toc}{subsection}{Accumulation and discount}

\[a(t)=(1+i)^t=(1-d)^{-t}\]

\[a^{-1}(t)=(1+i)^{-t}=(1-d)^t=v^t\]

\hypertarget{homework}{%
\section{Homework}\label{homework}}

\hypertarget{week-1}{%
\subsection*{Week 1}\label{week-1}}
\addcontentsline{toc}{subsection}{Week 1}

\hypertarget{problem-1}{%
\subsubsection*{Problem 1}\label{problem-1}}
\addcontentsline{toc}{subsubsection}{Problem 1}

John invests \(X\) in a fund growing in accordance with the accumulation function implied by the amount function
\[A(t)=4t^2+8t+4.\]
Edna invests \(X\) in another fund growing in accordance with the accumulation function implied by the amount function \[A(t)=4t^2+2.\]
When does Edna's investment \emph{exceed} John's?

\hypertarget{problem-2}{%
\subsubsection*{Problem 2}\label{problem-2}}
\addcontentsline{toc}{subsubsection}{Problem 2}

What deposit made today will provide for a payment of \(\$1000\) in 1 year and \(\$2000\) in 3 years, if the effective rate of interest is \(7.5\%\)?

\hypertarget{problem-3}{%
\subsubsection*{Problem 3}\label{problem-3}}
\addcontentsline{toc}{subsubsection}{Problem 3}

Company \(X\) received the approval to start no more than two projects in the current calendar year.
Three different projects were recommended, each of which requires an investment of 800 to be made at the beginning of the year.

The cash flows for each of the three projects are as follows:

\begin{table}

\caption{\label{tab:unnamed-chunk-1}The cash flows of the three projects.}
\centering
\begin{tabular}[t]{r|r|r|r}
\hline
End of year & Project A & Project B & Project C\\
\hline
1 & 500 & 500 & 500\\
\hline
2 & 500 & 300 & 250\\
\hline
3 & -175 & -175 & -175\\
\hline
4 & 100 & 150 & 200\\
\hline
5 & 0 & 200 & 200\\
\hline
\end{tabular}
\end{table}

The company uses an annual effective interest rate of \(10\%\) to discount its cash flows.

\hypertarget{week-2}{%
\subsection*{Week 2}\label{week-2}}
\addcontentsline{toc}{subsection}{Week 2}

  \bibliography{\_reference.bib}

\end{document}
